\documentclass[a4paper,12pt,titlepage,finall]{article}

\usepackage[T1,T2A]{fontenc}     
\usepackage[utf8x]{inputenc}     
\usepackage[russian]{babel}      
\usepackage{tikz}                
\usepackage{pgfplots}            
\usepackage{geometry}		 
\usepackage{indentfirst}         
\usepackage{multirow}            

\geometry{a4paper,left=30mm,top=30mm,bottom=30mm,right=30mm}

\setcounter{secnumdepth}{0}      
\usepgfplotslibrary{fillbetween} 

\begin{document}

% Титульный лист
\begin{titlepage}
    \begin{center}
	{\small \sc Московский государственный университет \\имени М.~В.~Ломоносова\\
	Факультет вычислительной математики и кибернетики\\}
	\vfill
	{\Large \sc Отчет по заданию №1}\\
	~\\
	{\large \bf <<Методы сортировки>>}\\ 
	~\\
	{\large \bf Вариант 6}
    \end{center}
    \begin{flushright}
	\vfill {Выполнил:\\
	студент 104 группы\\
	Карнаухов~М.~А.\\
	~\\
	Преподаватель:\\
	Гуляев~Д.~А.}
    \end{flushright}
    \begin{center}
	\vfill
	{\small Москва\\2026}
    \end{center}
\end{titlepage}

\tableofcontents
\newpage

\section{Постановка задачи}

В данной работе требуется реализовать два метода сортировки массива чисел и провести их экспериментальное сравнение. Сравниваются следующие алгоритмы:
\begin{itemize}
    \item Insertion Sort (сортировка вставками)
    \item Merge Sort (сортировка слиянием)
\end{itemize}
Элементы массивов — целые числа. Сортировка проводится по возрастанию.

\newpage

\section{Результаты экспериментов}

Для оценки эффективности алгоритмов измерялось количество сравнений элементов и количество перемещений для каждого алгоритма. Эксперимент проводился на массивах трёх типов:
\begin{itemize}
    \item случайный массив,
    \item отсортированный массив,
    \item обратно отсортированный массив.
\end{itemize}

\begin{table}[h!]
\centering
\begin{tabular}{|c|c|c|c|c|c|}
\hline
\textbf{n} & \textbf{Type} & \textbf{Ins. Moves} & \textbf{Ins. Comps} & \textbf{Merge Moves} & \textbf{Merge Comps} \\
\hline
\multirow{3}{*}{10} & Random & 43 & 29 & 68 & 23 \\
                     & Sorted & 18 & 9  & 68 & 19 \\
                     & Reverse& 63 & 45 & 68 & 15 \\
\hline
\multirow{3}{*}{100} & Random & 2\,483 & 2\,382 & 1\,344 & 547 \\
                      & Sorted & 198    & 99    & 1\,344 & 356 \\
                      & Reverse& 5\,148 & 4\,950 & 1\,344 & 316 \\
\hline
\multirow{3}{*}{1000} & Random & 252\,372 & 251\,370 & 19\,952 & 8\,696 \\
                       & Sorted & 1\,998  & 999      & 19\,952 & 5\,044 \\
                       & Reverse& 501\,498 & 499\,500 & 19\,952 & 4\,932 \\
\hline
\multirow{3}{*}{10000} & Random & 25\,011\,026 & 25\,001\,022 & 267\,232 & 120\,363 \\
                        & Sorted & 19\,998       & 9\,999        & 267\,232 & 69\,008  \\
                        & Reverse& 50\,014\,998  & 49\,995\,000   & 267\,232 & 64\,608  \\
\hline
\end{tabular}
\caption{Результаты работы сортировок Insertion Sort и Merge Sort для различных размеров массивов}
\end{table}

\subsection{Анализ результатов}

\begin{itemize}
    \item Insertion Sort демонстрирует квадратичную зависимость операций от размера массива. На случайных и обратно отсортированных массивах количество сравнений и перемещений растет почти пропорционально $O(n^2)$. На отсортированном массиве количество операций минимально, что соответствует лучшему случаю $O(n)$.
    \item Merge Sort показывает почти одинаковое количество операций на всех типах массивов, что подтверждает теоретическую оценку $O(n \log n)$.
    \item Сравнение показывает, что Merge Sort эффективнее на больших массивах, а Insertion Sort может быть применим для маленьких или почти отсортированных массивов.
\end{itemize}

\newpage

\section{Структура программы и спецификация функций}

Программа включает следующие функции:

\begin{itemize}
    \item \texttt{void insertionSort(int n, int a[])} — сортировка массива вставками, подсчет сравнений и перемещений.
    \item \texttt{void mergeSort(int arr[], int left, int right)} — сортировка массива слиянием, подсчет сравнений и перемещений.
    \item \texttt{void merge(int arr[], int left, int mid, int right)} — вспомогательная функция для слияния подмассивов.
    \item \texttt{void generateRandom(int arr[], int n, int maxVal)} — генерация случайного массива.
    \item \texttt{void generateSorted(int arr[], int n)} — генерация уже отсортированного массива.
    \item \texttt{void generateReverse(int arr[], int n)} — генерация обратно отсортированного массива.
\end{itemize}

\newpage

\section{Отладка программы, тестирование функций}

Тестирование проводилось на трёх типах массивов: случайный, отсортированный и обратно отсортированный. Для каждого теста проверялась корректность сортировки и фиксировались показатели сравнений и перемещений.

\newpage

\section{Анализ допущенных ошибок}

На этапе отладки были выявлены следующие моменты:
\begin{itemize}
    \item Ошибки при подсчете копирований в функции merge — исправлены.
    \item Некорректная индексация подмассивов при слиянии — исправлена.
    \item Проверка работы алгоритмов на различных типах массивов позволила убедиться в корректности подсчета операций.
\end{itemize}

\end{document}
